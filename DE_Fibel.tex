%%  ************    Chipforge's Fibel   *******************************
%%
%%  Organisation:   Chipforge
%%                  Germany / European Union
%%
%%  Profile:        Chipforge focus on fine System-on-Chip Cores in
%%                  Verilog HDL Code which are easy understandable and
%%                  adjustable. For further information see
%%                          www.chipforge.org
%%                  there are projects from small cores up to PCBs, too.
%%
%%  File:           Fibel/DE_fibel.tex
%%
%%  Purpose:        Top Level File DE (deutsch)
%%
%%  ************    LaTeX   *******************************************
%%
%%  ///////////////////////////////////////////////////////////////////
%%
%%  Copyright (c)   2019, 2020 by
%%                  chipforge <fibel@nospam.chipforge.org>
%%
%%  Permission is granted to copy, distribute and/or modify this
%%  document under the terms of the GNU Free Documentation License,
%%  Version 1.3 or any later version published by the Free Software
%%  Foundation; with no Invariant Sections, no Front-Cover Texts, and
%%  no Back-Cover Texts.
%%
%%   (__)   A copy of the license is included in the section entitled
%%   oo )   "GNU Free Documentation License".
%%   /_/|   http://www.gnu.org/licenses/fdl-1.3.html
%%
%%  ///////////////////////////////////////////////////////////////////
\documentclass[10pt,a4paper]{report}
\usepackage[utf8]{inputenc}
\usepackage[english]{babel}
%\usepackage{amsmath}
%\usepackage{amsfonts}
\usepackage{amssymb}
%\usepackage{gensymb}
%\usepackage{graphicx}
\usepackage[digital,srcmeas,semicon]{circdia}
%\usepackage[dvipsnames]{xcolor}
\usepackage[left=2cm,right=2cm,top=2cm,bottom=2cm]{geometry}

\title{Fibel}
\author{chipforge \texttt{<fibel@nospam.chipforge.org>}}
\date{\today}

\begin{document}

\maketitle
\setlength{\parindent}{0pt} % get rid of annoying indents

\begin{abstract}
\begin{quote}
Permission is granted to copy, distribute and/or modify this
document under the terms of the GNU Free Documentation License,
Version 1.3 or any later version published by the Free Software
Foundation; with no Invariant Sections, no Front-Cover Texts, and
no Back-Cover Texts.

A copy of the license is included in the section entitled
"GNU Free Documentation License".

http://www.gnu.org/licenses/fdl-1.3.html
\end{quote}
\end{abstract}

\clearpage
\tableofcontents
\clearpage

%%  -------------------------------------------------------------------
%%                  PART 1
%%  -------------------------------------------------------------------

%   Einfuehrung Digitaltechnik
\input{part-1/DE_part-1.tex}

%%  -------------------------------------------------------------------
%%                  PART 2
%%  -------------------------------------------------------------------

%   Coding Guide lines
\input{part-2/DE_part-2.tex}

%%  -------------------------------------------------------------------
%%                  PART 3
%%  -------------------------------------------------------------------

%   Schaltungstechnik
\input{part-3/DE_part-3.tex}

%%  -------------------------------------------------------------------
%%                  PART 4
%%  -------------------------------------------------------------------

%   Systems-on-Chip
%%  ************    Chipforge's Fibel   *******************************
%%
%%  Organisation:   Chipforge
%%                  Germany / European Union
%%
%%  Profile:        Chipforge focus on fine System-on-Chip Cores in
%%                  Verilog HDL Code which are easy understandable and
%%                  adjustable. For further information see
%%                          www.chipforge.org
%%                  there are projects from small cores up to PCBs, too.
%%
%%  File:           Fibel/part-4/DE_part-4.tex
%%
%%  Purpose:        Part Level File DE (deutsch)
%%
%%  ************    LaTeX   *******************************************
%%
%%  ///////////////////////////////////////////////////////////////////
%%
%%  Copyright (c)   2019, 2020 by
%%                  chipforge <fibel@nospam.chipforge.org>
%%
%%  Permission is granted to copy, distribute and/or modify this
%%  document under the terms of the GNU Free Documentation License,
%%  Version 1.3 or any later version published by the Free Software
%%  Foundation; with no Invariant Sections, no Front-Cover Texts, and
%%  no Back-Cover Texts.
%%
%%   (__)   A copy of the license is included in the section entitled
%%   oo )   "GNU Free Documentation License".
%%   /_/|   http://www.gnu.org/licenses/fdl-1.3.html
%%
%%  ///////////////////////////////////////////////////////////////////
\part{System-on-Chip}

%%  -------------------------------------------------------------------
%%                  TOPICS
%%  -------------------------------------------------------------------

Offene Punkte:
\begin{itemize}
    \item{SoC}
    \item{Interfaces / Bussysteme}
    \item{I2C, SPI}
    \item{Wishbone}
\end{itemize}


%%  -------------------------------------------------------------------
%%                  PART 5
%%  -------------------------------------------------------------------

%   DfT & Test
\input{part-5/DE_part-5.tex}

%%  -------------------------------------------------------------------
%%                  PART 6
%%  -------------------------------------------------------------------

%   Test bench & Simulation
\input{part-6/DE_part-6.tex}

%%  -------------------------------------------------------------------
%%                  PART 7
%%  -------------------------------------------------------------------

%   Tool Box
%%  ************    Chipforge's Fibel   *******************************
%%
%%  Organisation:   Chipforge
%%                  Germany / European Union
%%
%%  Profile:        Chipforge focus on fine System-on-Chip Cores in
%%                  Verilog HDL Code which are easy understandable and
%%                  adjustable. For further information see
%%                          www.chipforge.org
%%                  there are projects from small cores up to PCBs, too.
%%
%%  File:           Fibel/part-7/DE_part-7.tex
%%
%%  Purpose:        Part Level File DE (deutsch)
%%
%%  ************    LaTeX   *******************************************
%%
%%  ///////////////////////////////////////////////////////////////////
%%
%%  Copyright (c)   2019, 2020 by
%%                  chipforge <fibel@nospam.chipforge.org>
%%
%%  Permission is granted to copy, distribute and/or modify this
%%  document under the terms of the GNU Free Documentation License,
%%  Version 1.3 or any later version published by the Free Software
%%  Foundation; with no Invariant Sections, no Front-Cover Texts, and
%%  no Back-Cover Texts.
%%
%%   (__)   A copy of the license is included in the section entitled
%%   oo )   "GNU Free Documentation License".
%%   /_/|   http://www.gnu.org/licenses/fdl-1.3.html
%%
%%  ///////////////////////////////////////////////////////////////////
\part{Vorgehen}

%%  -------------------------------------------------------------------
%%                  TOPICS
%%  -------------------------------------------------------------------

Offene Punkte:

\begin{itemize}
    \item{gute Spezifikation, gute Planung}
    \item{mit den richtigen Open Source Tools zu Silizium}
    \item{Synthese einer Netzliste}
    \item{Place and Route, Clock Tree}
    \item{Setup- and Hold-Zeiten, STA}
    \item{GDS II}
\end{itemize}



%%  -------------------------------------------------------------------
%%                  APPENDIX
%%  -------------------------------------------------------------------

%   additional stuff
\input{appendix/DE_appendix.tex}

\end{document}
