%%  ************    Chipforge's Fibel   *******************************
%%
%%  Organisation:   Chipforge
%%                  Germany / European Union
%%
%%  Profile:        Chipforge focus on fine System-on-Chip Cores in
%%                  Verilog HDL Code which are easy understandable and
%%                  adjustable. For further information see
%%                          www.chipforge.org
%%                  there are projects from small cores up to PCBs, too.
%%
%%  File:           Fibel/part-3/section-ASRST/block-ASRST.tex
%%
%%  Purpose:        Code Example for ASRST
%%
%%  ************    LaTeX with circdia.sty package      ***************
%%
%%  ///////////////////////////////////////////////////////////////////
%%
%%  Copyright (c)   2019, 2020 by
%%                  chipforge <fibel@nospam.chipforge.org>
%%
%%  Permission is granted to copy, distribute and/or modify this
%%  document under the terms of the GNU Free Documentation License,
%%  Version 1.3 or any later version published by the Free Software
%%  Foundation; with no Invariant Sections, no Front-Cover Texts, and
%%  no Back-Cover Texts.
%%
%%   (__)   A copy of the license is included in the section entitled
%%   oo )   "GNU Free Documentation License".
%%   /_/|   http://www.gnu.org/licenses/fdl-1.3.html
%%
%%  ///////////////////////////////////////////////////////////////////
\begin{center}
    \begin{figure}[h]
        \begin{verbatim}
module asyncrst (
    output logic sreset;
    input logic clock;
    input logic aresetn);

    parameter NSTAGES = 3;

logic [NSTAGES-1:0] shiftreg;

// clocked process
always @ (posedge clock, negedge aresetn)
begin
    if (!aresetn)
        // reset active
        shiftreg <= {NSTAGES{1'b1}};
    else
        // shift in reset release value from MSB while clocked
        shiftreg <= {1'b0, shiftreg[NSTAGES-1:1]};
end

// shiftregister LSB used as output
assign sreset = shiftreg[0];

endmodule
        \end{verbatim}
    \end{figure}
    SystemVerilog 2012
\end{center}
