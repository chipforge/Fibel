%%  ************    Chipforge's Fibel   *******************************
%%
%%  Organisation:   Chipforge
%%                  Germany / European Union
%%
%%  Profile:        Chipforge focus on fine System-on-Chip Cores in
%%                  Verilog HDL Code which are easy understandable and
%%                  adjustable. For further information see
%%                          www.chipforge.org
%%                  there are projects from small cores up to PCBs, too.
%%
%%  File:           Fibel/part-3/section-SIWTM/DE_section-SIWTM.tex
%%
%%  Purpose:        Topic SIWTM File DE (deutsch)
%%
%%  ************    LaTeX   *******************************************
%%
%%  ///////////////////////////////////////////////////////////////////
%%
%%  Copyright (c)   2019, 2020 by
%%                  chipforge <fibel@nospam.chipforge.org>
%%
%%  Permission is granted to copy, distribute and/or modify this
%%  document under the terms of the GNU Free Documentation License,
%%  Version 1.3 or any later version published by the Free Software
%%  Foundation; with no Invariant Sections, no Front-Cover Texts, and
%%  no Back-Cover Texts.
%%
%%   (__)   A copy of the license is included in the section entitled
%%   oo )   "GNU Free Documentation License".
%%   /_/|   http://www.gnu.org/licenses/fdl-1.3.html
%%
%%  ///////////////////////////////////////////////////////////////////
\section{Einsynchronisieren mit einem Testmultiplexer} \label{part-3:SIWTM}

- an Blockübergängen [siehe ?????] müssen hereinkommende Signal oftmals als asynchron betrachtet werden
- ebenso bei Übergängen von Signalen aus einer fremden Clockdomain in die blockeigene Clockdomain sind hereinkommende Signale nicht synchron

- d.h. diese Signale sind wieder einsynchronisiert werden

%%  ************    Chipforge's Fibel   *******************************
%%
%%  Organisation:   Chipforge
%%                  Germany / European Union
%%
%%  Profile:        Chipforge focus on fine System-on-Chip Cores in
%%                  Verilog HDL Code which are easy understandable and
%%                  adjustable. For further information see
%%                          www.chipforge.org
%%                  there are projects from small cores up to PCBs, too.
%%
%%  File:           Fibel/part-3/section-SIWTM/figure-SIWTM.tex
%%
%%  Purpose:        Figure File for SIWTM
%%
%%  ************    LaTeX with circdia.sty package      ***************
%%
%%  ///////////////////////////////////////////////////////////////////
%%
%%  Copyright (c)   2019, 2020 by
%%                  chipforge <fibel@nospam.chipforge.org>
%%
%%  Permission is granted to copy, distribute and/or modify this
%%  document under the terms of the GNU Free Documentation License,
%%  Version 1.3 or any later version published by the Free Software
%%  Foundation; with no Invariant Sections, no Front-Cover Texts, and
%%  no Back-Cover Texts.
%%
%%   (__)   A copy of the license is included in the section entitled
%%   oo )   "GNU Free Documentation License".
%%   /_/|   http://www.gnu.org/licenses/fdl-1.3.html
%%
%%  ///////////////////////////////////////////////////////////////////
\begin{center}
    \begin{figure}[h]
        \begin{center}
            \begin{circuitdiagram}{44}{15}
            \pin{1}{1}{L}{clock}    % pin clock
            \wire{2}{1}{31}{1}
            \junct{11}{1}
            \junct{21}{1}
            \pin{1}{7}{L}{tmode}    % pin tmode
            \pin{1}{9}{L}{async}    % pin async
            \usgate
            \decoder{mux21}{6}{9}{R}{}{MUX2}
            \wire{11}{1}{11}{7}
            \wire{11}{7}{12}{7}
            \flipflop{d}{16}{7}{R}{}{DFF}
            \wire{21}{1}{21}{7}
            \wire{21}{7}{22}{7}
            \flipflop{d}{26}{7}{R}{}{DFF}
            \wire{31}{1}{31}{7}
            \wire{31}{7}{32}{7}
            \flipflop{d}{36}{7}{R}{}{DFF}
            \wire{10}{9}{12}{9}
            \wire{20}{9}{22}{9}
            \wire{30}{9}{32}{9}
            \wire{40}{9}{42}{9}
            \junct{41}{9}
            \wire{41}{9}{41}{14}
            \wire{1}{14}{41}{14}
            \wire{1}{11}{1}{14}
            \wire{1}{11}{2}{11}
            \pin{43}{9}{R}{sync}    % pin sync
            \end{circuitdiagram}
        \end{center}
    \end{figure}
\end{center}


- dazu wird wieder die bereits in [?????] vorgestellte Synchronisierungstechnik verwendet
- um die Testmassnahmen in der eigenen Clockdomain zu unterstützen, sollten hereinkommende Signale auf definierte, bekannte Werte gesetzt werden
- dies erfolgt hier mit einem Multiplexer

%%  ************    Chipforge's Fibel   *******************************
%%
%%  Organisation:   Chipforge
%%                  Germany / European Union
%%
%%  Profile:        Chipforge focus on fine System-on-Chip Cores in
%%                  Verilog HDL Code which are easy understandable and
%%                  adjustable. For further information see
%%                          www.chipforge.org
%%                  there are projects from small cores up to PCBs, too.
%%
%%  File:           Fibel/part-3/section-SIWTM/block-SIWTM.tex
%%
%%  Purpose:        Code Example for SIWTM
%%
%%  ************    LaTeX with circdia.sty package      ***************
%%
%%  ///////////////////////////////////////////////////////////////////
%%
%%  Copyright (c)   2019, 2020 by
%%                  chipforge <fibel@nospam.chipforge.org>
%%
%%  Permission is granted to copy, distribute and/or modify this
%%  document under the terms of the GNU Free Documentation License,
%%  Version 1.3 or any later version published by the Free Software
%%  Foundation; with no Invariant Sections, no Front-Cover Texts, and
%%  no Back-Cover Texts.
%%
%%   (__)   A copy of the license is included in the section entitled
%%   oo )   "GNU Free Documentation License".
%%   /_/|   http://www.gnu.org/licenses/fdl-1.3.html
%%
%%  ///////////////////////////////////////////////////////////////////
\begin{center}
    \begin{figure}[h]
        \begin{verbatim}
module asyncintm (
    input logic async;
    output logic sync;
    input logic tmode;
    input logic clock;
    input logic sreset);

    parameter NSTAGES = 3;

logic muxout;
logic [NSTAGES-1:0] shiftreg;

// test feedback multiplexer
assign muxout = (tmode)? shiftreg[0]: async;

// clocked process
always @ (posedge clock, posedge sreset)
begin
    if (sreset)
        // reset active
        shiftreg <= {NSTAGES{1'b0}};
    else
        // shift in muxed signal from MSB while clocked
        shiftreg <= {muxout, shiftreg[NSTAGES-1:1]};
end

// shiftregister LSB used as output
assign sync = shiftreg[0];

endmodule
        \end{verbatim}
    \end{figure}
    SystemVerilog 2012
\end{center}


- bei aktiven Testmodesignal (tmode) wird dazu das Ausgangssignal wieder in die Schiebereigsterkette zurück gefüht
~> die Clockdomain ist damit von dem asynchronen Signal abgekoppelt
\clearpage
